\documentclass[12pt,a4paper]{article}
\usepackage[utf8x]{inputenc}
\usepackage{ucs}
\usepackage[spanish]{babel}
\usepackage{amsmath}
\usepackage{amsfonts}
\usepackage{amssymb}
\usepackage{makeidx}
\usepackage{graphicx}
\usepackage{hyperref}
\hypersetup{
	colorlinks,
	citecolor=black,
	filecolor=black,
	linkcolor=red,
	urlcolor=black
}
\usepackage[width=17.00cm, height=23.00cm]{geometry}
\author{Jorge Yero Salazar - C412\\ Jose Diego Menendez Del Cueto - C412\\ Jose Ariel Romero Acosta - C412}
\title{Reporte de entrega del proyecto final de la
	asignatura Complementos de Compilación}
\date{Curso: 2018-2019}
\begin{document}
	\maketitle
	\tableofcontents
	\section{Introducción}
		En el presente reporte se tratan los aspectos fundamentales de la implementación de un compilador de COOL. En la sección 2 se plantean como obtener y utilizar el proyecto, se analizan todos los requerimientos y cual es la funcionalidad del compilador. La sección 3 expone cual es la arquitectura del compilador implementado, se plantean las diferentes fases desde que se recibe el código COOL hasta obtener el código ensamblador. En la sección 4 se trata sobre el analizador léxico y el sintáctico y la utilización de una herramienta generadora de analizadores lexicográficos y sintácticos así como las modificaciones que se le hizo a la gramática de COOL. La sección 5 trata sobre el analizador semántico cuales fueron las fases por las que se pasa y el por qué de ellas. La sección 6 sobre la generación de código intermedio, la explicación cual código fue utilizado que no es uno estandarizado sino uno definido por nosotros. La sección 7 sobre la generación de código ensamblador en cuyo caso la arquitectura es MIPS. En la sección 8 se trata una de las fases mas interesantes pues se trata la estrategia seguida para probar el compilador y se puede apreciar su funcionamiento.
	\section{Requisitos y uso del proyecto}
	\section{Arquitectura}
	\section{Analizador Léxico y Analizador Sintáctico}
	\section{Analizador Semántico}
	\section{Generación de Código Intermedio}
	\section{Generación de Código MIPS}
	\section{Probar del compilador}
\end{document}