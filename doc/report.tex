\documentclass[12pt,a4paper]{article}
\usepackage[utf8x]{inputenc}
\usepackage{ucs}
\usepackage[spanish]{babel}
\usepackage{amsmath}
\usepackage{amsfonts}
\usepackage{amssymb}
\usepackage{makeidx}
\usepackage{graphicx}
\usepackage{hyperref}
\usepackage{listings}
\usepackage{multicol}
\usepackage{float}
\hypersetup{
	colorlinks,
	citecolor=black,
	filecolor=black,
	linkcolor=red,
	urlcolor=black
}
\usepackage[width=17.00cm, height=23.00cm]{geometry}
\author{Jorge Yero Salazar - C412\\ Jose Diego Menendez Del Cueto - C412\\ Jose Ariel Romero Acosta - C412}
\title{Reporte de entrega del proyecto final de la
	asignatura Complementos de Compilación}
\date{Curso: 2018-2019}
\lstset{
	language=bash,
	breaklines=true,
	tabsize=1
}
\begin{document}
	\maketitle
	\tableofcontents
	\section{Introducción}
		En el presente reporte se tratan los aspectos fundamentales de la implementación de un compilador de COOL. En la sección 2 se plantean como obtener y utilizar el proyecto, se analizan todos los requerimientos y cual es la funcionalidad del compilador. La sección 3 expone cual es la arquitectura del compilador implementado, se plantean las diferentes fases desde que se recibe el código COOL hasta obtener el código ensamblador. En la sección 4 se trata sobre el analizador léxico y el sintáctico y la utilización de una herramienta generadora de analizadores lexicográficos y sintácticos así como las modificaciones que se le hizo a la gramática de COOL. La sección 5 trata sobre el analizador semántico cuales fueron las fases por las que se pasa y el por qué de ellas. La sección 6 sobre la generación de código intermedio, la explicación cual código fue utilizado que no es uno estandarizado sino uno definido por nosotros. La sección 7 sobre la generación de código ensamblador en cuyo caso la arquitectura es MIPS. En la sección 8 se trata una de las fases mas interesantes pues se trata la estrategia seguida para probar el compilador y se puede apreciar su funcionamiento.
	\section{Requisitos y uso del proyecto}
		La presente implementación de un compilador del lenguaje COOL fue echa en
		.Net Core 2.1. El único requisito para poder utilizar el compilador es tener instalado la versión 2.1 o superior. El proyecto puede ser descargado ejecutando 
		\begin{lstlisting}
			git clone https://github.com/matcom-compilers-2019/cool-compiler-supercool.git
		\end{lstlisting}
		y para conocer mas información acerca de los requerimientos y el uso debe leerse \url{https://github.com/matcom-compilers-2019/cool-compiler-supercool/doc/Readme.md}
	\section{Arquitectura}
		\begin{figure}[H]
			\centering
			\includegraphics[scale=0.5]{Informe/Arquitectura.png}
			\caption{Arquitectura del compilador}
		\end{figure}
	\section{Analizador Léxico y Analizador Sintáctico}
	\section{Analizador Semántico}
		\subsection{Boxing y Unboxing}
			Uno de los aspectos fundamentales fue detectar en los lugares donde era necesario hacer boxing y unboxing.
			\begin{center}
				Lugares donde se hizo boxing 
			\end{center} 
			\begin{multicols}{2}
				\begin{itemize}
					\item paso de argumentos
					\item retorno de funciones de tipos por referencia
					\item distpatch
					\item expr0 en un case
				\end{itemize}
			\end{multicols}
			\begin{center}
				Lugares donde se hizo unboxing 
			\end{center} 
			\begin{multicols}{2}
				\begin{itemize}
					\item en las expresiones del case que fueran de tipos por valor
					\item después de haber ejecutado al copy de un tipo por valor
				\end{itemize}
			\end{multicols}
	\section{Generación de Código Intermedio}
		En la etapa de generación de código intermedio se generó a un código intermedio que no es está estandarizado sino a uno establecido por nosotros el cual estaba mas cercano a COOL que a ensamblador, esto nos limitó las cosas que se podían hacer en el código intermedio y en ocasiones se tuvo que generar código mips directamente de la salida del AST como en el caso de la expresión case, si el código intermedio generado por nosotros estuviese a un nivel mas bajo se podría haber resuelto esta expresión en esta etapa sin necesidad de bajarlo directamente a la próxima. Para la generación de código se realiza un visitor sobre el AST que da como salida el analizador semántico y se traducen los nodos del AST a uno o mas nodos de código intermedio, los nodos que se decidió bajar directamente del AST a la generación de código ensamblador lo que se realizó fue crear un nodo en esta etapa prácticamente con la misma signatura que en el AST y se asignaron sus parámetros.
	\section{Generación de Código MIPS}
		Para la generación del código ensamblador, en este caso el código mips lo que se realiza es un visitor sobre el árbol de codigo intermedio que se generó como se explicó en la sección anterior, luego se traduce cada nodo de código intermedio a código mips. Para facilitar esta etapa se realizo un Helper con macros implementadas que fueron de gran utilidad como 
		Push, Pop, Call, PrintInt, PrintString, Add, Sub, Mul, Div, entre otras.
		\subsection{Representación de tipos}
			Uno de los aspectos importantes a destacar es como se representaron los tipos en mips, para ello se dividen los tipos en tipos por valor y tipos por referencia. Los tipos por referencia son representados de la siguiente forma
			\begin{multicols}{2}
				\begin{table}[H]
					\centering
					\begin{tabular}{|c|c|}
						\hline
						índice & valor \\\hline
						-4 & type\_info \\\hline
						0 & atributo$_1$ \\\hline
						\vdots & \vdots \\\hline
						$4*(n-1)$ & atributo$_n$ \\\hline 
					\end{tabular} 
					\caption{Representación de un tipo por referencia}		
				\end{table}
				\begin{table}[H]
					\centering
					\begin{tabular}{|c|c|}
						\hline
						índice & valor \\\hline
						0 & type\_name\_ref \\\hline
						4 & allocate\_size \\\hline
						8 & virtua\_table\_ref \\\hline
					\end{tabular}
					\caption{Representación del type\_info}		
				\end{table}
			\end{multicols}
		\begin{table}[H]
			\centering
			\begin{tabular}{|c|c|}
				\hline
				índice & valor \\\hline
				0 & función$_1$\_ref \\\hline
				\vdots & \vdots \\\hline
				$4 * (n - 1)$ & función$_n$\_ref \\\hline
			\end{tabular}
			\caption{Representación de la virtual\_table}		
		\end{table}
	\section{Probar del compilador}
		El compilador viene con un conjunto de test para probar todas las funcionalidades, estos pueden probarse estando en la raíz del proyecto y ejecutando 
		\begin{lstlisting}
			make tests
		\end{lstlisting}
		los test solo son capaces de ejecutarse en Linux 64 bits y Windows 10 64 bits. Es necesario tener instalado spim para poder correr el código mips, en Windows 10 es necesario tener instalado \textit{Windows Subsystem For Linux}. Hay test que ejecutan hacen la ejecución completa del compilador, ademas ejecutan el código mips, para saber si la salida es correcta se compara utilizando ademas del compilador implementado otro compilador para y se comparan las salidas de ambos.
\end{document}